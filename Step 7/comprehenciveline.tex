Element	Description
<div>	Generic container used for layout and grouping content
<p>	Paragraph text
<h1> to <h6>	Headings, from most important (<h1>) to least (<h6>)
<ul>	Unordered list (used with <li>)
<ol>	Ordered list (used with <li>)
<li>	List item (must be inside <ul> or <ol>)
<section>	Defines a section in a document
<article>	Represents a self-contained piece of content
<nav>	Defines navigation links
<header>	Introductory content, usually contains logo, nav, etc.
<footer>	Footer for a document or section
<form>	Used to collect user input
<address> –contact For  information

<blockquote> – For long quotations

<pre> – Preformatted text (respects spaces and line breaks)

<fieldset> – Groups related form elements

<legend> – Title for a <fieldset>

<hr> – Horizontal rule (divider)

<noscript> – Content to display if JavaScript is disabled

<details> – Expandable/collapsible content container

<summary> – Title for <details> element

<figure> – Groups images with captions

<figcaption> – Caption for <figure>

<aside> – Sidebar or content tangentially related

<canvas> – Container for graphics/drawings via JavaScript

<output> – Result of a calculation or user action

<dialog> – Dialog box or popup

<iframe> – Inline frame (though layout depends on styling, it's block-level by default)





Inline element list (most commonly used first)

<span> — Generic inline container for styling (like inline <div>)

<a> — Anchor element (for hyperlinks)

<img> — Embeds an image

<strong> — Bold text with semantic meaning (importance)

<em> — Italic text with semantic meaning (emphasis)

<br> — Line break

<label> — Labels form inputs

<input> — Input field (text, checkbox, radio, etc.)

<select> — Dropdown list

<textarea> — Multi-line text input (technically behaves inline-block)

<button> — Clickable button (inline-block)

<abbr> — Abbreviation with a tooltip

<code> — Displays code snippets

<small> — Smaller text

<i> — Italic text (for styling, no semantic meaning)

<b> — Bold text (for styling only)

<u> — Underlined text

<sup> — Superscript text (e.g., x²)

<sub> — Subscript text (e.g., H₂O)

<time> — Machine-readable time/date
<cite>	Cites a creative work (e.g., book, article, movie)
<q>	Short quotation (renders with quotes)
<kbd>	Keyboard input
<samp>	Sample output from a program
<var>	Variable name in programming
<mark>	Highlighted or marked text
<bdi>	Isolates text direction (Bi-Directional Isolation)
<bdo>	Overrides text direction
<data>	Links content to machine-readable value
<dfn>	Term being defined
<ruby>	For East Asian typography (annotation of text)
<rt> / <rp>	Used inside <ruby> for pronunciation
<wbr>	Word break opportunity (hint to browser for line breaking)
<output>	Display result of a calculation or user interaction
<progress>	Progress bar (inline-block)
<meter>	Scalar measurement within a range (e.g., disk usage)


